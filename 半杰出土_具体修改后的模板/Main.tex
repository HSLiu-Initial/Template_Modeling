% !Mode:: "TeX:UTF-8"
%!TEX program  = xelatex

%\documentclass{cumcmthesis}
\documentclass[withoutpreface,bwprint]{cumcmthesis} %去掉封面与编号页

\usepackage{url}
\usepackage{booktabs}

\usepackage[colorlinks,
             linkcolor=blue,       %%修改此处为你想要的颜色
            anchorcolor=blue,  %%修改此处为你想要的颜色
            citecolor=blue,        %%修改此处为你想要的颜色,例如修改blue为red
             ]{hyperref}

%感觉这个要跟你们商量一下
%\usepackage[compress]{cite}


%我们是电子版论文所以我把这个注释掉了=-=
%\tihao{C}
%\baominghao{4321}
%\schoolname{中山大学}
%\membera{A}
%\memberb{B}
%\memberc{C}
%\supervisor{老师}
%\yearinput{2019}
%\monthinput{09}
%\dayinput{15}

\title{ }


\begin{document}

	\maketitle
 	\begin{abstract}


	\keywords{}
	\end{abstract}




%目录
\tableofcontents
\newpage

\section{问题重述}
	\subsection{背景知识}

	\subsection{相关数据}

	\subsection{具体问题}


\section{问题的分析}
	\subsection{研究现状综述}

	\subsection{对问题的总体分析和解题思路}

	\subsection{对具体问题的分析和对策}

\section{模型的假设}

\section{名词解释和说明}

	\subsection{名词解释}
	\subsection{符号说明}

\begin{table}[!htp]
\centering
\caption{本文所用符号的相关说明与解释}\label{Tab:1}
\begin{tabular}{cc}
 \toprule[1.5pt]
%\makebox[0.3\textwidth][c]{符号}	&  \makebox[0.4\textwidth][c]{意义} \\
序号 &	符号 &   意义 \\ 
\midrule

\bottomrule[1.5pt]
\end{tabular}
\end{table}

\section{模型的建立与解决}
%在这一部分每个问题都必须有一个独立的subsection去进行说明!!

\section{误差分析与灵敏度分析}
	\subsection{误差分析}

	\subsection{灵敏度分析}

\section{模型的评价与推广}
	\subsection{模型的评价}

	\subsection{模型的推广}

\section{模型的改进}

%参考文献


\bibliography{BJCT}
\bibliographystyle{is-unsrt}

%\begin{thebibliography}{99}%宽度9
% \bibitem{bib:one} 韩中庚,数学建模方法及其应用,北京:高等教育出版社,2009。 
% \bibitem{bib:two}姜启源,谢金星,叶俊,数学模型,北京:高等教育出版社,2003。 
% \bibitem{bib:three}龚纯,王正林,精通 MATLAB 最优化计算,北京:电子工业出版社,2012。 
% \bibitem{bib:four}卓金武,李必文,魏永生,秦健,MATLAB 在数学建模中的应用,北京:北京 航空航天大学出版社,2014。
%\end{thebibliography}

\newpage
%附录
\appendix
\section{主程序--matlab 源程序}
\begin{lstlisting}[language=matlab]
function [dist,xiane,xinyuzhi] = clac()
info_renwu = load('renwu_information.txt');
row = size(info_renwu,1);
col = size(info_renwu,2);
dist = zeros(row,1);
xiane  = zeros(row,1);
xinyuzhi = zeros(row,1);
dingjia = info_renwu(:,3);
for i = 1:row
[dist(i),xiane(i),xinyuzhi(i)]=variable(info_renwu,i); 
end

 \end{lstlisting}

\section{计算距离--matlab 源程序}
\begin{lstlisting}[language=matlab]
function dis = distance(wei1,jing1,wei2,jing2)
R = 6370;
%dis = asin(sqrt(sin((wei1-wei2)/2)^2+cos(wei1)*cos(wei2)*sin((jing1-jing2)/2)^2));
%dis = dis*2*R;
dis = 2*pi*R/360*((wei1-wei2)^2+(jing1-jing2)^2*cos((wei1+wei2)/2)^2)^0.5;

\section{聚类--matlab 源程序}
\begin{lstlisting}[language=matlab]
function [dist,xiane,xinyuzhi] = variable(info_renwu,i)
info_huiyuan = load('huiyuan_location.txt');
data_xiane = info_huiyuan(:,3);
data_xinyu = info_huiyuan(:,4);
data_wei = info_huiyuan(:,1);
data_jing = info_huiyuan(:,2);
num = size(data_jing,1);
data_distance = zeros(num,1);
sum = 0;
for j = 1:num
data_distance(j) = distance(info_renwu(i,1),info_renwu(i,2),info_huiyuan(j,1),info_huiyuan(j,2));
end
[a,b] = sort(data_distance);
first = b(1:30,:);
for count = 1:30
sum = sum + data_distance(first(count));
end
dist = sum/30;
sum = 0;
for count = 1:30
sum = sum + data_xiane(first(count));
end
xiane = sum/30;
sum = 0;
for count = 1:30
sum = sum + data_xinyu(first(count));
end
xinyuzhi = sum/30;

\end{lstlisting}






\end{document} 